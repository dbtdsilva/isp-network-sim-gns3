\documentclass[11pt,a4paper]{report}
\usepackage[english]{babel}
\usepackage[utf8]{inputenc}
\usepackage[T1]{fontenc}
\usepackage{glossaries}
\usepackage{graphicx}
\usepackage{hyperref}
\usepackage{wrapfig}
\usepackage{float}
\usepackage{natbib}
\usepackage{listings}
\usepackage{caption}
\usepackage{subcaption}
\newcommand{\HRule}{\rule{\linewidth}{0.5mm}}
\setlength\parindent{0pt} % Removes all indentation from paragraphs	 
%% Acronimos
%\newglossaryentry{sqlite}
%{
%  name=SQLite,
%  description={é um sistema de gestão de base de dados}
%}
%\newacronym{aes}{AES}{Advanced Encryption Standard}
%\newacronym{sha}{SHA}{Secure Hash Algorithm}
%\makeglossaries
%% Fim da introdução do Acronimos

\let\olditemize\itemize
\renewcommand{\itemize}{
  \olditemize
  \setlength{\itemsep}{1pt}
  \setlength{\parskip}{0pt}
  \setlength{\parsep}{0pt}
}

\title{\textbf{Internet Service Provider ARA Project} \\Arquitectura de Redes Avançada \\Universidade de Aveiro}
\author{Diogo Silva 60337 \and Eduardo Sousa 68633 }

\begin{document}
\begin{titlepage}
\begin{center}
\HRule \\[0.4cm]
{ \huge \bfseries Internet Service Provider ARA Project \\[0.4cm] }
\HRule \\[1.5cm]
\textsc{\LARGE Universidade de Aveiro}\\[1.5cm]
\textsc{}\\[1.5cm]
\textsc{Diogo Silva 60337 \\Eduardo 68633 }
\end{center}
\end{titlepage}
\maketitle
\tableofcontents

\chapter{Basic Mechanisms and BGP}

\section{Internal BGP \& OSPF Redistribution}
\#EDUARDO
\section{External BGP}
\#EDUARDO
\section{Private AS}
\#EDUARDO

\section{Routing Constraints}

Neste projecto todas as restrições de routing apresentadas a seguir foram efectuadas usando route-map para efectuar a respectiva regra, ou negar a rota, ou aumentar a local preference da rede anunciada no iBGP.

\subsection{Internet Traffic}

``IP traffic towards Internet should be preferably routed via ISP S (Lisboa).''
\newline

\begin{lstlisting}
ip router bgp 9.345
network yolo
 test abc
 asd aw 0400
!

\end{lstlisting}

\subsection{Net L1 and Net L2 Preferences}

``IP traffic towards netL1 and netL2, should be preferably routed via Porto from Aveiro, and via Lisboa from Faro.''

asd

\subsection{SIP Proxy 2 Traffic}

``IP traffic for remote SIP proxy 2 (to network netS1) should be routed only via Lisboa using the direct peering link to ISP S.''
\newline

asd

\subsection{Non-Transit ISP-X}



\section{Changes for IPv6}
\#EDUARDO

\chapter{MPLS}
\#DIOGO
\section{MPLS Tunnel for SIP Traffic}
\section{MPLS VPN}


\chapter{VoIP SIP}
\#EDUARDO
\section{Internal Extensions}
\section{PTSN Calls Support}
\section{Forward to SIP Proxy 2}

\bibliographystyle{plain}
\bibliography{proj2}

\end{document}
